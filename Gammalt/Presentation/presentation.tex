%\documentclass[handout]{beamer}
\documentclass[•]{beamer}
\usepackage[latin1]{inputenc}
\usepackage{amsmath}
\usetheme{Luebeck}
\setbeamertemplate{footline}[frame number] 
\setbeamertemplate{navigation symbols}{} 

\newtheorem{proposition}{Proposition}
\newtheorem{exercise}{Exercise}
\theoremstyle{remark}
\newtheorem{remark}[theorem]{Remark}

\mathchardef\sa="303A
\newcommand{\esup}{\mathop{{\rm ess\,sup}}\limits}
\let\Eqnarray=\eqnarray
\renewcommand{\eqnarray}{\arraycolsep=0.1675em \Eqnarray}

\newcommand{\lag}{\mathcal{L}}
\newcommand{\e}{\mathrm{e}}

% enumabc
%
\makeatletter
\newcommand{\enumabc}
	{\expandafter\def\csname the\@enumctr \endcsname{\alph{\@enumctr}}%
	 \expandafter\def\csname label\@enumctr \endcsname
		{\rm(\csname the\@enumctr \endcsname)}}%
\newcommand{\enuminc}[1]{\addtocounter{\@enumctr}{#1}}
\makeatother
%
\subtitle{Transmission properties in a short biased quantum wire}
\title{TFYA17 Project}
\author{Patrik Hallsj\"{o}, Felix Faber}
\date{}
\AtBeginSection[]
{
  \begin{frame}
    \frametitle{Table of Contents}
    \tableofcontents[currentsection]
  \end{frame}
}
\AtBeginSubsection[]
{
  \begin{frame}
    \frametitle{Table of Contents}
    \tableofcontents[currentsection,currentsubsection]
  \end{frame}
}

\begin{document}
\begin{frame}
\titlepage
\end{frame}
\begin{frame}
\tableofcontents
\end{frame}
\section{Introduction}
\begin{frame}[shrink=10]\frametitle{Introduction}
\begin{block}

In this Bachelor's thesis the following question was answered:\\
Does the inequality posed in the article Klyachko et al [2008] cover the real
part of the Bloch surface of a 3D quantum system when used as in Kochen
and Specker [1967]?
\end{block}
\pause
\begin{remark}
However to understand the questions, some introduction is needed.
\end{remark}
\end{frame}

\end{document}